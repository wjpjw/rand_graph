\documentclass[a4paper,11pt]{article}
\usepackage{cv}

\usepackage[utf8]{inputenc}
\usepackage{listings}
\usepackage{color}
\usepackage{amsthm}
\usepackage[english]{babel}
 
\newtheorem{theorem}{Theorem}

\definecolor{codegreen}{rgb}{0,0.6,0}
\definecolor{codegray}{rgb}{0.5,0.5,0.5}
\definecolor{codepurple}{rgb}{0.58,0,0.82}
\definecolor{backcolour}{rgb}{0.95,0.95,0.92}
\lstdefinestyle{mystyle}{
    backgroundcolor=\color{backcolour},   
    commentstyle=\color{codegreen},
    keywordstyle=\color{magenta},
    numberstyle=\tiny\color{codegray},
    stringstyle=\color{codepurple},
    basicstyle=\footnotesize,
    breakatwhitespace=false,         
    breaklines=true,                 
    captionpos=b,                    
    keepspaces=true,                 
    numbers=left,                    
    numbersep=5pt,                  
    showspaces=false,                
    showstringspaces=false,
    showtabs=false,                  
    tabsize=2
}
\lstset{style=mystyle}

\name{\hspace{0.3cm}Jipeng, Sam}
\info{CSE-6010 Assignment 3: \\[0.2cm]
      Graph Analytics Part 2}
      
\bibliography{rjhpubs}
\begin{document}
\maketitle
\thispagestyle{empty}


\section{Literature Search And Expected Results, Sam}

Herrera and Zufiria use a random walk algorithm to generate a scale-free network in [1].  They also outline  a more traditional approach to generating a scale free network in the beginning of the paper.  That approach is outlined here:

\subsection{Initializing}
\begin{enumerate}
\item{Initialize the graph with $m_0$ nodes}
\item{Create an edge between each of the initial nodes}
\end{enumerate}

In our implementation, $m_0$ was chosen to be 5.

\subsection{Filling out Rest of Nodes}
For each remaining node to be generated, the node most be connected to $m$ nodes.  In our implementation, $m$ was also chosen to be 5.

Determine which node to connect the new node to was done using a probabilistic distribution where:

\begin{equation} \nonumber
p_{i}=\dfrac{k_{i}}{\sum_{j=1}^{n}k_{j}}
\end{equation}

Where $k_{i}$ represents the number of edges connected to node $i$, and $n$ is the set of nodes already in the graph that are not already connected to node $i$.  We draw from this distribution $m$ times, each time updating the set $n$ to no longer include the last edge that node $i$ was connected to.



 
\section{Graph Analysis, Jipeng}
The result:
\begin{enumerate}
\item N:10,   diameter:0.560740
\item N:100,  diameter:1.524168
\item N:1000, diameter:1.805032
\end{enumerate}

The result shows that all scale-free graphs are connected graphs. The diameter is also satisfies:$(1-e)\frac{logn}{loglogn} \le diam(G) \le (1+e)\frac{logn}{loglogn}$.


\begin{thebibliography}{9}
\bibitem{latexcompanion}
  Carlos Herrera and Pedro J. Zufiria.
  \textit{Generating Scale-free Networks with Adjustable
Clustering Coefficient Via Random Walks}.
  http://arxiv.org/pdf/1105.3347.pdf
\end{thebibliography}  

\end{document}
